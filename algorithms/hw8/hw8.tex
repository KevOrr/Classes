% Created 2017-06-27 Tue 17:25
% Intended LaTeX compiler: pdflatex
\documentclass[11pt]{article}
\usepackage[utf8]{inputenc}
\usepackage[T1]{fontenc}
\usepackage{graphicx}
\usepackage{grffile}
\usepackage{longtable}
\usepackage{wrapfig}
\usepackage{rotating}
\usepackage[normalem]{ulem}
\usepackage{amsmath}
\usepackage{textcomp}
\usepackage{amssymb}
\usepackage{capt-of}
\usepackage{hyperref}
\usepackage[boxed, linesnumbered, commentsnumbered, noend, noline]{algorithm2e}
\usepackage{mathtools}
\usepackage{amsmath}
\DeclarePairedDelimiter\ceil{\lceil}{\rceil}
\DeclarePairedDelimiter\floor{\lfloor}{\rfloor}
\usepackage[margin=1.25in]{geometry}
\author{Kevin Orr}
\date{June 25, 2017}
\title{Homework 8}
\hypersetup{
 pdfauthor={Kevin Orr},
 pdftitle={Homework 8},
 pdfkeywords={},
 pdfsubject={},
 pdfcreator={Emacs 25.2.2 (Org mode 9.0.9)}, 
 pdflang={English}}
\begin{document}

\maketitle
\begin{enumerate}
\item The inner-most for loop runs \(deg(V)\) times in the worst case, with each iteration taking
O(1) time, and the call to \texttt{getNeighbors} taking O(\(deg(v_j)\)) time. The outer for loop also
runs \(deg(v)\) times, with at most O(1) overhead per iteration in addition to the inner for
loop, and the call to \texttt{getNeighbors} taking O(\(deg(v_i)\)) time.

\begin{equation*}
\begin{align*}
  T_{for}(v_i, v_j) &= \rm O(deg(v_i)) + \sum_{v_x\in N(V_i)} \left[ \rm O(deg(v_j)) + \sum_{v_y\in N(v_j)} \rm O(1) \right] \\
                    &= \rm O(deg(v_i)) + \sum_{v_x\in N(V_i)} \left[ \rm O(deg(v_j)) + \rm O(deg(v_j)) \right] \\
                    &= \rm O(deg(v_i)) + \sum_{v_x\in N(V_i)} \rm O(deg(v_j)) \\
                    &= \rm O(deg(v_i)) + \rm O(deg(v_i)) \rm O(deg(v_j)) \\
                    &= \rm O(deg(v_i) + deg(v_i) deg(v_j)) \\
                    &= \rm O(deg(v_i) deg(v_j))
\end{align*}
\end{equation*}
\linebreak

\item \begin{equation*}
\begin{align*}
  T(m, n) &= \sum_{v_i \in V} \sum_{v_j \in V} T_{for}(v_i, v_j) \\
          &= \sum_{v_i \in V} \sum_{v_j \in V} \rm O(deg(v_i) deg(v_j)) \\
          &= \sum_{v_i \in V} \sum_{v_j \in V} \rm O(deg(v_i)) \rm O(deg(v_j)) \\
          &= \sum_{v_i \in V} \left[ \rm O(deg(v_i)) \sum_{v_j \in V} \rm O(deg(v_j)) \right] \\
          &= \sum_{v_i \in V} deg(v_i) \rm O(2m) \\
          &= \rm O(2m) \sum_{v_i \in V} deg(v_i) \\
          &= \rm O(2m) \rm O(2m) \\
          &= \rm O(4m^2) \\
          &= \rm O(m^2)
\end{align*}
\end{equation*}

\item This is the same time complexity as 2. The function will iterate over each pair of vertices no less than
once if the graph is fully connected, and no more than once since it will skip over already-visited pairs,
making the total time complexity \(\rm O(m^2)\).
\end{enumerate}
\end{document}
