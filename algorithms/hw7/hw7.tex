% Created 2017-06-22 Thu 15:30
% Intended LaTeX compiler: pdflatex
\documentclass[11pt]{article}
\usepackage[utf8]{inputenc}
\usepackage[T1]{fontenc}
\usepackage{graphicx}
\usepackage{grffile}
\usepackage{longtable}
\usepackage{wrapfig}
\usepackage{rotating}
\usepackage[normalem]{ulem}
\usepackage{amsmath}
\usepackage{textcomp}
\usepackage{amssymb}
\usepackage{capt-of}
\usepackage{hyperref}
\usepackage[boxed, linesnumbered, commentsnumbered, noend, noline]{algorithm2e}
\usepackage{mathtools}
\DeclarePairedDelimiter\ceil{\lceil}{\rceil}
\DeclarePairedDelimiter\floor{\lfloor}{\rfloor}
\usepackage[margin=1.25in]{geometry}
\author{Kevin Orr}
\date{06/21/2017}
\title{Homework 7}
\hypersetup{
 pdfauthor={Kevin Orr},
 pdftitle={Homework 7},
 pdfkeywords={},
 pdfsubject={},
 pdfcreator={Emacs 25.2.2 (Org mode 9.0.7)}, 
 pdflang={English}}
\begin{document}

\maketitle
\begin{enumerate}
\item Take for example \(\phi(12)\). The current pseudocode will find that 2 divides 12, so it will
then compute \(\phi(2)\) and \(\phi(6)\). In the call to \(\phi(6)\), it will find that 2 again divides
6, and call \(\phi(2)\) and \(\phi(3)\). In this sense, when the argument has duplicate primes in its
prime factorization, it will call \(\phi\) on each of those duplicates. This means that memoization
would be a good candidate for this algorithm.

\item Since the function only takes one positive integral argument, a growing array would be the correct
data structure to use for memoization.

\item Since all positive integers are co-prime with all least 1 (1 is co-prime with itself since it has
\emph{no} primes in its prime factorization), then \(\phi(n)\) is positive for all positive \(n\). A sentinel
value should be chosen that is not in this co-domain of \(\phi\). A good choice is 0.
\pagebreak

\item Below is a memoized version of the provided \texttt{EulerPhi} according to (2) and (3).
\begin{function}
  \DontPrintSemicolon
  \KwIn{$n$: a positive integer}
  \KwOut{$\phi(n)$}
  \TitleOfAlgo{eulerPhi}

  $memos = $ [\,] \;
  \Return memoPhi(n, $memos$)\;
\end{function}

\begin{function}
  \DontPrintSemicolon
  \KwIn{$n$: a positive integer}
  \KwIn{$memos$: array of memoized results}
  \KwOut{$\phi(n)$}
  \TitleOfAlgo{memoPhi}

  \;
  \tcp{Extend the array and fill the new elements with 0's:}
  \For{$i = $ \rm{length}$(memos)$ \KwTo $n-1$}{
    $memos[i] = 0$\;
  }

  \;
  \tcp{Check if function has been called before}
  \If{$memos[n-1] \ne 0$}{
    \Return $memos[n-1]$\;
  }

  \;
  \For{$a = 2$ \rm{to} $\floor*{\sqrt{n}}$}{
    \If{$a$ \rm{divides} $n$}{
      $b = n/a$\;
      $g$ = gcd$(a,b)$\;
      \Return memoPhi$(a)~ \cdot$ memoPhi$(b) \cdot g/$memoPhi$(g)$\;
    }
  }

  \;
  \Return $n-1$\;
\end{function}
\end{enumerate}
\end{document}
